\section{R�sum� des besoins}
%%%%%%%%%%%%%%%%%%%%%%%%%%%%
Le client souhaite pouvoir faire disposer � ses �l�ves d'un jeu les initiant au fonctionnement du syst�me immunitaire dans le corps humain.
\\
\\
Une certaine polyvalence du jeu est requise : celui-ci doit �tre utilisable par diff�rent niveaux scolaires, et proposer aussi bien une initiation qu'un approfondissement aux notions abord�es.

\section{Description du probl�me et de sa solution}
%%%%%%%%%%%%%%%%%%%%%%%%%%%%%%%%%%%%%%%%%%%%%%%%%%%
Le jeu se pr�sente sous la forme d'un combat spatial en 2 dimensions, entre les �l�ments du syst�me immunitaire r�pr�sent�s comme des vaisseaux spatiaux. Le joueur contr�lera l'immunit� pour combattre les agents pathog�nes.
\\
\\
On veut permettre un apprentissage au cours duquel les connaissances sont valid�es par �tapes au fil des niveaux.
\\
\\
Pour ce faire, le d�roulement suivant sera respect� :
%%
\begin{itemize}
\item L'utilisateur lance le jeu
\item L'utilisateur clique sur jouer
\item L'utilisateur choisi son niveau de difficult�
\item L'utilisateur lance un niveau
\item L'utilisateur termine un niveau
\item Une fiche informations est disponible pour plus de d�tails sur les nouvelles notions introduites par le niveau d�bloqu�\\
\end{itemize}
%%
%%
Si le joueur se retrouve bloqu� par un manque de connaissances - sans lesquelles il ne peut savoir comment �liminer l'ennemi - il aura acc�s � une fiche d�taill�e sur l'agent pathog�ne � combattre.
\\
\\
La fiche sera affich�e sous la forme d'un briefing de mission, tout en restant compl�te et non abusive d'un point de vue scientifique.