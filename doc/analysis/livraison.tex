Constitution des livraisons 	
\begin{itemize}
	\item le cahier des charges
	\item le document de conception
	\item un carnet de r�cits d'utilisation
	\item le journal de d�veloppement
	\item un manuel d'utilisation et un tutoriel
\end{itemize}
organisation  des livraisons
\begin{itemize}
	\item un r�pertoire {\tt src} contenant les scripts de l'application
	\item un r�pertoire {\tt lib} contenant les librairies externes utilis�es pour des fonctionnalit�s de l'application
	\item un r�pertoire {\tt data} contenant les diff�rentes donn�es (images, textes, \ldots)
	\item un r�pertoire {\tt doc} contenant  les diff�rentes documentations aff�rentes au projet.
\end{itemize}
\section{Premi�re version : menu}
Acc�der au menu du jeu
\begin{itemize}
	\item cliquer sur jouer
	\item d�sactiver la musique
	\item d�sactiver les effets sonores
	\item quitter le jeu
\end{itemize}
\section{Deuxi�me version : gestion des niveaux}
G�rer ses niveaux d�bloqu�s
\begin{itemize}
	\item acc�der aux niveaux
	\item acc�der aux fiches
	\item charger son profil joueur
\end{itemize}
\section{Troisi�me version : gestion des combats avec bact�ries}
G�rer les combats avec bact�ries
\begin{itemize}
	\item syst�me de combat
	\item d�cors
	\item ennemis
	\item boss
	\item m�thode d'�limination
\end{itemize}
\section{Quatri�me version : gestion des combats avec virus}
G�rer les combats avec virus
\begin{itemize}
	\item ennemis
	\item boss
	\item m�thode d'�limination
\end{itemize}
