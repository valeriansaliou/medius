\section{Conclusion}
Medius sera un jeu aux graphismes attractifs (couleurs dynamiques, interface utilisateur accessible) et au rythme de jeu soutenu. Ceci permettrait d'accrocher le joueur et lui permettre d'aller plus loin dans son initiation au fonctionnement du syst�me immunitaire humain.
Les joueurs les plus avanc�s, d�sireux d'en apprendre plus ou de devenir meilleur au jeu, pourront r�gler la difficult� jusqu'� 3.
\\
\\
Medius sera donc adapt� � tout type de public. Nous certifions ses bases scientifiques correctes, qui s'appuient sur nos cours de Terminale S SVT sur le syst�me immunitaire.

\section{Perspectives}
Le jeu pourrait �tre utilis� dans des cas r�els d'aide � l'apprentissage du fonctionnement du syst�me immunitaire par des Terminales S SVT (un chapitre de leur programme y est d�di�, d�crivant tr�s pr�cis�ment son fonctionnement).
\\
\\
De m�me, il serait int�ressant d'initier les coll�giens de mani�re ludique aux notions de base du syst�me immunitaire, ce qui serait une bonne occasion de leur r�v�ler toute la dangerosit� du VIH, par lequel les populations jeunes sont particuli�rement concern�es. Il faudrait pour cela ajouter un mode "Impossible" ou l'on combat sp�cialement le virus du SIDA, � savoir VIH.